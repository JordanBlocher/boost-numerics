\documentclass[12pt]{article}
 
\usepackage[text={6in,8.1in},centering]{geometry}

\usepackage{enumerate}
\usepackage{amsmath,amsthm,amssymb}
\usepackage{mathrsfs} % to use mathscr fonts

\usepackage{epstopdf}
\usepackage{caption,subcaption}
\usepackage{pstricks}
\usepackage{pst-solides3d}
\usepackage{pstricks-add}
\usepackage{graphicx}
\usepackage{pst-tree}
\usepackage{pst-poly}
\usepackage{calc,ifthen}
\usepackage{float}\usepackage{multicol}
\usepackage{multirow}
\usepackage{array}
\usepackage{longtable}
\usepackage{fancyhdr}
\usepackage{algorithmicx,algpseudocode}
\usepackage{changepage}
\usepackage{color}
\usepackage{listings}
\usepackage{fancyvrb}
\usepackage{verbatim,moreverb}
\usepackage{courier}

\lstset{ %
language=C++,               
basicstyle=\footnotesize,
numbers=left,                  
numberstyle=\tiny,     
stepnumber=1,         
numbersep=5pt,         
backgroundcolor=\color{white},  
showspaces=false,               
showstringspaces=false,         
showtabs=false,                 
columns=fullflexible,
frame=single,          
tabsize=2,          
captionpos=b,       
extendedchars=true,
xleftmargin=17pt,
framexleftmargin=17pt,
framexrightmargin=17pt,
framexbottommargin=4pt,
breaklines=true,       
breakatwhitespace=false, 
escapeinside={\%*}{*)}       
}

\newenvironment{block}{\begin{adjustwidth}{1.5cm}{1.5cm}\noindent}{\end{adjustwidth}}

\newtheorem{proposition}{Proposition}[section]
\newtheorem{theorem}{Theorem}[section]
\newtheorem{lemma}{Lemma}[section]
\newtheorem{corollary}{Corollary}[section]
\theoremstyle{definition}
\newtheorem{definition}{Definition}[section]

 
\def\R{\mbox{$\mathbb R$}}
\def\Q{\mbox{$\mathbb Q$}}
\def\Z{\mbox{$\mathbb Z$}}
\def\N{\mbox{$\mathbb N$}}
\def\C{\mbox{$\mathbb C$}}
\def\Sym{\operatorname{Sym}}
\def\lcm{\operatorname{lcm}}
\def\adj{\operatorname{adj}}
\def\inc{\operatorname{inc}}
\def\Geom{\operatorname{\cal G}}
\def\ker{\operatorname{ker}}
\def\kernel{\operatorname{ker}}
\def\automorphism{\operatorname{Aut}}
\def\endomorphism{\operatorname{End}}
\def\inner{\operatorname{Inn}}
\def\outer{\operatorname{Out}}
\def\crossing{\operatorname{cr}}
\def\cent{\textcent}
\def\n{\\ \vspace{1.7mm}}
\def\diam{\operatorname{diam}}
 
\def\verbatimtabsize{4\relax}
\def\listingoffset{1em}
\def\listinglabel#1{\llap{\tiny\it\the#1}\hskip\listingoffset\relax}
\def\mylisting#1{{\fontsize{10}{11}\selectfont \listinginput[1]{1}{#1}}}
\def\myoutput#1{{\fontsize{9}{9.2}\selectfont\verbatimtabinput{#1}}}



\renewcommand{\emptyset}{\O}
 
 
\newcounter{ZZZ}
\newcounter{XXX}
\newcounter{XX}
 
\headsep25pt\headheight20pt
 
 
\pagestyle{fancyplain}
\lhead{\fancyplain{}{\small\bfseries Blocher, Jordan}}
\rhead{\fancyplain{}{\small\bfseries Math 701}}
\cfoot{\ \hfill\tiny\sl Draft printed on \today}
 
 
\setlength{\extrarowheight}{2.5pt} % defines the extra space in tables
 
\begin{document}

\subsection*{Wave Equation}

\begin{enumerate}
% p. 1
\item Solve
\begin{center}
$\begin{array}{lclll}
v_t-v_x&=& 0 && x\in(0,1)\\
v(x,0)&=&\sin^{40}\pi x && x\in[0,1]\\
v(0,t)&=& v(1,t), && t\geq 0\\
\end{array}$
\end{center}
Use the FTFS difference scheme:
\[u_k^{n+1}=u_k^n-R(u_{k+1}^n -u_k^n)\] along with an appropriate treatment of the periodic boundary conditions. Use $M=20$, and $\Delta t=0.04 \ (R=a\Delta t/\Delta x =-0.8)$. Plot solutions at times $t=0.0, 0.12, 0.2 $ and $0.8$.

\item Repeat with $M=100$ and $\Delta t=0.008$

% p. 2
\item Repeat using the Crank-Nicholson scheme:
\[-\frac{R}{4}u_{k-1}^{n+1}+u_k^{n+1}+\frac{R}{4}u_{k+1}^{n+1}=\frac{R}{4}u_{k-1}^n-\frac{R}{4}u_{k+1}^n\]

Use $M=20$, and $\Delta t=0.04 \ (R=a\Delta t/\Delta x =-0.8)$. Plot solutions at times $t=0.0, 0.12, 0.2 $ and $0.8$.
\graphicspath{ {4/} }

\item Repeat with $M=100$ and $\Delta t=0.008$. Plot solutions at times $t=0.0,5.0,10.0$ and $20.0$

% p. 3
\item Use the difference schemes:
\[u_k^{n+1}=u_k^n -R\delta_-u_k^n+r\delta^2u_k^n, \ a>0\]
and 
\[u_k^{n+1}=u_k^n -\frac{R}{2}\delta_0u_k^n+\frac{R^2}{2}\delta^2u_k^n+r\delta^2u_k^n\]
to approximate the solutions to the initial-boundary-value problem.
\begin{center}
$\begin{array}{rllllll}
v_t+v_x&=&\nu v_{xx}, &&& t\\
v(0,t)&=&v(1,t)\ = \ 0, &&& t>0\\
v(x,0)&=&3\sin 4\pi x, &&& x\in[0,1]\\
\end{array}$
\end{center}
Let $\nu=0.00001$ and use $M=100, \Delta t=0.005$ and plot solutions at times $t=0.1,0.5,1.0$ and $2.0$.

\end{enumerate}

\end{document}

