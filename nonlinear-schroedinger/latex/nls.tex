\documentclass[12pt]{article}
 
\usepackage[text={6in,8.1in},centering]{geometry}

\usepackage{enumerate}
\usepackage{amsmath,amsthm,amssymb}
\usepackage{mathrsfs} % to use mathscr fonts

\usepackage{epstopdf}
\usepackage{caption,subcaption}
\usepackage{pstricks}
\usepackage{pst-solides3d}
\usepackage{pstricks-add}
\usepackage{graphicx}
\usepackage{pst-tree}
\usepackage{pst-poly}
\usepackage{calc,ifthen}
\usepackage{float}\usepackage{multicol}
\usepackage{multirow}
\usepackage{array}
\usepackage{longtable}
\usepackage{fancyhdr}
\usepackage{algorithmicx,algpseudocode}
\usepackage{changepage}
\usepackage{color}
\usepackage{listings}
\usepackage{fancyvrb}
\usepackage{verbatim,moreverb}
\usepackage{courier}

\graphicspath{ {../} }
\graphicspath{ {../} }

\lstset{ %
language=C++,               
basicstyle=\footnotesize,
numbers=left,                  
numberstyle=\tiny,     
stepnumber=1,         
numbersep=5pt,         
backgroundcolor=\color{white},  
showspaces=false,               
showstringspaces=false,         
showtabs=false,                 
columns=fullflexible,
frame=single,          
tabsize=2,          
captionpos=b,       
extendedchars=true,
xleftmargin=17pt,
framexleftmargin=17pt,
framexrightmargin=17pt,
framexbottommargin=4pt,
breaklines=true,       
breakatwhitespace=false, 
escapeinside={\%*}{*)}       
}

\newenvironment{block}{\begin{adjustwidth}{1.5cm}{1.5cm}\noindent}{\end{adjustwidth}}

\newtheorem{proposition}{Proposition}[section]
\newtheorem{theorem}{Theorem}[section]
\newtheorem{lemma}{Lemma}[section]
\newtheorem{corollary}{Corollary}[section]
\theoremstyle{definition}
\newtheorem{definition}{Definition}[section]

 
\def\R{\mbox{$\mathbb R$}}
\def\Q{\mbox{$\mathbb Q$}}
\def\Z{\mbox{$\mathbb Z$}}
\def\N{\mbox{$\mathbb N$}}
\def\C{\mbox{$\mathbb C$}}
\def\Sym{\operatorname{Sym}}
\def\lcm{\operatorname{lcm}}
\def\adj{\operatorname{adj}}
\def\inc{\operatorname{inc}}
\def\Geom{\operatorname{\cal G}}
\def\ker{\operatorname{ker}}
\def\kernel{\operatorname{ker}}
\def\automorphism{\operatorname{Aut}}
\def\endomorphism{\operatorname{End}}
\def\inner{\operatorname{Inn}}
\def\outer{\operatorname{Out}}
\def\crossing{\operatorname{cr}}
\def\cent{\textcent}
\def\n{\\ \vspace{1.7mm}}
\def\diam{\operatorname{diam}}
 
\def\verbatimtabsize{4\relax}
\def\listingoffset{1em}
\def\listinglabel#1{\llap{\tiny\it\the#1}\hskip\listingoffset\relax}
\def\mylisting#1{{\fontsize{10}{11}\selectfont \listinginput[1]{1}{#1}}}
\def\myoutput#1{{\fontsize{9}{9.2}\selectfont\verbatimtabinput{#1}}}



\renewcommand{\emptyset}{\O}
 
 
\newcounter{ZZZ}
\newcounter{XXX}
\newcounter{XX}
 
\headsep25pt\headheight20pt
 
 
\pagestyle{fancyplain}
\lhead{\fancyplain{}{\small\bfseries Blocher, Jordan}}
\rhead{\fancyplain{}{\small\bfseries Math 701}}
\cfoot{\ \hfill\tiny\sl Draft printed on \today}
 
 
\setlength{\extrarowheight}{2.5pt} % defines the extra space in tables
 
\begin{document}


\subsection*{Nonlinear Schrodinger Equation}

\begin{enumerate}
\item Consider the nonlinear Schrodinger equation
\begin{center}
$\left \{\begin{array}{rcll}
iv_t&=& v_{xx}+2\vert v\vert^2v &\text{ for } (x,t)\in\mathbb{R}\times(0,T)\\
v(x,0)&=&f(x) & \text{ for } x\in\mathbb{R}\\
\end{array}\right.$
\end{center}

Let $v$ be the solution to the nonlinear Schrodinger equation with $L$-periodic boundary conditions. Let $\Delta x=L/K$ and $\Delta t=T/N$ and consider the finite difference method for approximating $v$ on $[0,L]\times[0,T]$ given by
\begin{center}
$\left\{
\begin{array}{rcllrr}
u_k^{n+1}&=& u_k^{n-1}-\frac{2i\Delta t}{\Delta x^2}\delta^2u_k^n-4i\Delta t\vert u_k^n\vert^2u_k^n & \text{ for } n&=& 1,\cdots,N-1\\
&&&\text{ and } k&=&1,\cdots,K\\ 
u_k^1&=& u_k^0-\frac{i\Delta t}{\Delta x^2}\delta^2u_k^0-2i\Delta t\vert u_k^0\vert^2u_k^0 & \text{ for } k&=&1,\cdots,K\\
u_0^n&=&u_K^n\ \text{ and }\ u_{K+1}^n=U-1^n& \text{ for } n&=&0,\cdots,N\\
u_k^n&=&f(k\Delta x) & \text{ for } k&=&1,\cdots,K\\
\end{array}\right.$
\end{center}
For $L=10$ and $T=0.2$ choose $K$ and $N$ sufficiently large to approximate $v(5,0.2)$ to at least $3$ significant digits for
\[f(x)=2e^{-i\omega x}-i\cos{3\omega x}\]
where $\omega=\pi/5$.

\item Approximate $v(5,0.2)$ using the modified Crank-Nicolson Scheme. Include a bibliographic reference to the method as a comment in your source code.

\end{enumerate}

\end{document}

