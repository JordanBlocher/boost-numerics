\documentclass[12pt]{article}
 
\usepackage[text={6in,8.1in},centering]{geometry}

\usepackage{enumerate}
\usepackage{amsmath,amsthm,amssymb}
\usepackage{mathrsfs} % to use mathscr fonts

\usepackage{pstricks}
\usepackage{pst-solides3d}
\usepackage{pstricks-add}
\usepackage{graphicx}
\usepackage{pst-tree}
\usepackage{pst-poly}
\usepackage{calc,ifthen}
\usepackage{float}\usepackage{multicol}
\usepackage{multirow}
\usepackage{array}
\usepackage{longtable}
\usepackage{tikz}
\usepackage{tkz-berge}
\usepackage{fancyhdr}
\usepackage{algorithmicx,algpseudocode}
\usepackage{changepage}
\usepackage{color}
\usepackage{listings}
\usepackage{fancyvrb}
\usepackage{verbatim,moreverb}
\usepackage{courier}

\lstset{ %
language=C++,               
basicstyle=\footnotesize,
numbers=left,                  
numberstyle=\tiny,     
stepnumber=1,         
numbersep=5pt,         
backgroundcolor=\color{white},  
showspaces=false,               
showstringspaces=false,         
showtabs=false,                 
columns=fullflexible,
frame=single,          
tabsize=2,          
captionpos=b,       
extendedchars=true,
xleftmargin=17pt,
framexleftmargin=17pt,
framexrightmargin=17pt,
framexbottommargin=4pt,
breaklines=true,       
breakatwhitespace=false, 
escapeinside={\%*}{*)}       
}

\newenvironment{block}{\begin{adjustwidth}{1.5cm}{1.5cm}\noindent}{\end{adjustwidth}}

\newtheorem{proposition}{Proposition}[section]
\newtheorem{theorem}{Theorem}[section]
\newtheorem{lemma}{Lemma}[section]
\newtheorem{corollary}{Corollary}[section]
\theoremstyle{definition}
\newtheorem{definition}{Definition}[section]

 
\def\R{\mbox{$\mathbb R$}}
\def\Q{\mbox{$\mathbb Q$}}
\def\Z{\mbox{$\mathbb Z$}}
\def\N{\mbox{$\mathbb N$}}
\def\C{\mbox{$\mathbb C$}}
\def\Sym{\operatorname{Sym}}
\def\lcm{\operatorname{lcm}}
\def\adj{\operatorname{adj}}
\def\inc{\operatorname{inc}}
\def\Geom{\operatorname{\cal G}}
\def\ker{\operatorname{ker}}
\def\kernel{\operatorname{ker}}
\def\automorphism{\operatorname{Aut}}
\def\endomorphism{\operatorname{End}}
\def\inner{\operatorname{Inn}}
\def\outer{\operatorname{Out}}
\def\crossing{\operatorname{cr}}
\def\cent{\textcent}
\def\n{\\ \vspace{1.7mm}}
\def\diam{\operatorname{diam}}
 
\def\verbatimtabsize{4\relax}
\def\listingoffset{1em}
\def\listinglabel#1{\llap{\tiny\it\the#1}\hskip\listingoffset\relax}
\def\mylisting#1{{\fontsize{10}{11}\selectfont \listinginput[1]{1}{#1}}}
\def\myoutput#1{{\fontsize{9}{9.2}\selectfont\verbatimtabinput{#1}}}

\renewcommand{\emptyset}{\O}
 
 
\newcounter{ZZZ}
\newcounter{XXX}
\newcounter{XX}
 
\headsep25pt\headheight20pt
 
 
\pagestyle{fancyplain}
\lhead{\fancyplain{}{\small\bfseries Blocher, Jordan}}
\rhead{\fancyplain{}{\small\bfseries Math 701}}
\cfoot{\ \hfill\tiny\sl Draft printed on \today}
 
 
\setlength{\extrarowheight}{2.5pt} % defines the extra space in tables
 
\begin{document}

\subsection*{The QR Algorithm of Francis}
\noindent

\begin{enumerate}
% 1
\item Write computer subroutines or procedures for reducing a matrix to Hessenberg form via the equation \begin{center}$\begin{bmatrix}I&0\\0&U\end{bmatrix}\begin{bmatrix}B&C\\D&E\end{bmatrix}\begin{bmatrix}I&0\\0&U^*\end{bmatrix}=\begin{bmatrix}B&CU^*\\UD&UEU^*\end{bmatrix}$\end{center} and apply it to the matrix\begin{center}$\begin{bmatrix}&\begin{bmatrix}1\end{bmatrix}&\begin{bmatrix}2&3&4\end{bmatrix}&\\\\&\begin{bmatrix}4\\2\\4\end{bmatrix}&\begin{bmatrix}5&6&7\\1&5&0\\2&1&0\end{bmatrix}&\end{bmatrix}$\end{center}
% 2
\item Add computer subroutings or procedures for carrying out the basic $QR$-algorithm and reproduce the following results: The eigenvalues are $(2.1994,0.17645$, and $11.106)$. \begin{center}$A_{20}=\begin{bmatrix}11.106&-4.7403&3.9060&-4.0296\\0.&-3.8526&-0.68985&1.2559\\0.&-0.032156&3.5706&0.15706\\0.&0.&0.&0.17645\end{bmatrix}$\end{center}
% 3
\item Modify the computer subroutines or procedures so that they carry out the shifted $QR$-algorithm. Then reproduce the results from Example 3: The eigenvalues are $(3.5736, 0.17645, 11.106,$ and $-3.8556)$. Also, use your routines to find the eigenvalues $(343, 294,$ and $147\pm196i)$ of the matrix \begin{center}$\begin{bmatrix}190&66&-84&30\\66&303&42&-36\\336&-168&147&-112\\30&-36&28&291\end{bmatrix}$\end{center}

\end{enumerate}

\end{document}
